\documentclass[twoside,11pt]{article}

% Any additional packages needed should be included after jmlr2e.
% Note that jmlr2e.sty includes epsfig, amssymb, natbib and graphicx,
% and defines many common macros, such as 'proof' and 'example'.
%
% It also sets the bibliographystyle to plainnat; for more information on
% natbib citation styles, see the natbib documentation, a copy of which
% is archived at http://www.jmlr.org/format/natbib.pdf

\usepackage{jmlr2e}
\usepackage{bbm}
\usepackage{amsmath}

% Definitions of handy macros can go here
%%%%%%%%%%%%%%%%%%%%%%%%%%%%%%%%%%%%%%%%%%%%%%%%%%%%%%%%%%%%%%%%%%%%
% PACKAGES                                                          %
%%%%%%%%%%%%%%%%%%%%%%%%%%%%%%%%%%%%%%%%%%%%%%%%%%%%%%%%%%%%%%%%%%%%%


%%%%%%%%%%%%%%%%%%%%%%%%%%%%% begin additional packages [bottou]
\usepackage{fleqn}    %% for compatibility with amsmath fleqn 
\usepackage{booktabs} %% professional looking tables
\usepackage{multirow} %% multiple row cells in tables
%%%%%%%%%%%%%%%%%%%%%%%%%%%%% end additional packages [bottou]

%%%%%%%%%%%%%%%%%%%%%%%%%%%%% begin additional packages [bach]
\usepackage{subfig}
\usepackage{subfloat}
%%%%%%%%%%%%%%%%%%%%%%%%%%%%% end additional packages [bach]

\usepackage{color}
%\usepackage{epsf}
%\usepackage{makeidx}
\usepackage{ifthen}
\usepackage{eucal}

%% see ftp://tug.ctan.org/pub/tex-archive/macros/latex/required/graphics/grfguide.pdf
\usepackage{graphicx}
%\usepackage{psfrag}


%\bibpunct{[}{]}{,}{n}{,}{,}

%% see ftp://tug.ctan.org/pub/tex-archive/macros/latex/required/amslatex/math/amsldoc.pdf
%\usepackage[fleqn]{amsmath}
\usepackage{amssymb}
\usepackage{amsbsy}

%% see ftp://tug.ctan.org/pub/tex-archive/macros/latex/contrib/algorithms/algorithms.pdf
%% and ftp://tug.ctan.org/pub/tex-archive/macros/latex/contrib/algorithmicx/algorithmicx.pdf
\usepackage{algorithm}
\usepackage{algorithmicx}
%\usepackage{algmatlab} %% yom-tov
\usepackage{algpseudocode} %% loosli
%\usepackage{lscape} %% jegelka
%\usepackage{enumerate}
%\usepackage[boxed,vlined]{haffner/algorithm2e} %% haffner HACKED!
% \usepackage[ruled,vlined]{algorithm2e} %% haffner, raykar HACKED!

%% see
%% ftp://tug.ctan.org/pub/tex-archive/macros/latex/required/amslatex/classes/amsthdoc.pdf
\let\proof\relax
\let\endproof\relax
\usepackage{amsthm}
%\newtheorem{theorem}{Theorem}
%\newtheorem{definition}{Definition}
%\newtheorem{example}{Example}
%\newtheorem{assumption}{Assumption}
%\newtheorem{proposition}[theorem]{Proposition}
%\newtheorem{corollary}[theorem]{Corollary}
%\newtheorem{lemma}[theorem]{Lemma}
%\newtheorem{remark}{Remark}
%\newtheorem{application}{Application}
%\newtheorem{optimizationproblem}[example]{Optimization~Problem}
%\newtheorem{textunit}{}

% use \url{http://...} for urls.
\usepackage{url}

% this prints the index entries as margin notes (useful for debugging)
%\usepackage{showidx}  

% this makes captions a bit smaller
% usepackage{smallcaptions}

% for proper bold math, including bold greek letters; typesets in math-mode
% and not in \mathrm font
\usepackage{bm}

% package for including pdf pages from other sources
\usepackage{pdfpages}

\usepackage{titletoc}
%%%%%%%%%%%%%%%%%%%%%%%%%%%%%%%%%%%%%%%%%%%%%%%%%%%%%%%%%%%%%%%%%%%%%
% MACROS FOR SPECIFIC SYMBOLS, ETC.                                 %
%                                                                   %
% please have a look at that for the notation and tex macros        %
% you are going to use. this is provided for your convenience       %
% AND the consistency of the notation throughout the book           %
%                                                                   %
% PLEASE USE THESE MACROS WHEREVER POSSIBLE                         %
%%%%%%%%%%%%%%%%%%%%%%%%%%%%%%%%%%%%%%%%%%%%%%%%%%%%%%%%%%%%%%%%%%%%%




% %%%%%%%%%%%%%  VECTORS %%%%%%%%%%%%%%%%%%%
\newcommand{\bmt}[1]{\bm{#1}^T} % transpose of a vector
% INNER PRODUCT <x,y>
\newcommand{\ip}[2]{\langle {#1},\, {#2} \rangle}

% \va, \vb etc. type bold letters; notice exception at \vech -- coz. \vh is a
% predefined thing in latex; 
% additional commands include \vah -- read as vector a hat, 
% or \ah -- a hat and so on
\newcommand{\va}{\bm{a}}       \newcommand{\vah}{\hat{\bm{a}}}        \newcommand{\ah}{\hat{a}}
\newcommand{\vb}{\bm{b}}       \newcommand{\vbh}{\hat{\bm{b}}}        \newcommand{\bh}{\hat{b}}
\newcommand{\vc}{\bm{c}}       \newcommand{\vch}{\hat{\bm{c}}}        \newcommand{\ch}{\hat{c}}
\newcommand{\vd}{\bm{d}}       \newcommand{\vdh}{\hat{\bm{d}}}        \newcommand{\dhat}{\hat{d}}
\newcommand{\ve}{\bm{e}}       \newcommand{\veh}{\hat{\bm{e}}}        \newcommand{\eh}{\hat{e}}
\newcommand{\vf}{\bm{f}}       \newcommand{\vfh}{\hat{\bm{f}}}        \newcommand{\fh}{\hat{f}}
\newcommand{\vg}{\bm{g}}       \newcommand{\vgh}{\hat{\bm{g}}}        \newcommand{\gh}{\hat{g}}
\newcommand{\vech}{\bm{h}}     \newcommand{\vhh}{\hat{\bm{h}}}        \newcommand{\hh}{\hat{h}}
\newcommand{\vi}{\bm{i}}       \newcommand{\vih}{\hat{\bm{i}}}        \newcommand{\ih}{\hat{i}}
\newcommand{\vj}{\bm{j}}       \newcommand{\vjh}{\hat{\bm{j}}}        \newcommand{\jh}{\hat{j}}
\newcommand{\vk}{\bm{k}}       \newcommand{\vkh}{\hat{\bm{k}}}        \newcommand{\kh}{\hat{k}}
\newcommand{\vl}{\bm{l}}       \newcommand{\vlh}{\hat{\bm{l}}}        \newcommand{\lh}{\hat{l}}
\newcommand{\vm}{\bm{m}}       \newcommand{\vmh}{\hat{\bm{m}}}        \newcommand{\mh}{\hat{m}}
\newcommand{\vn}{\bm{n}}       \newcommand{\vnh}{\hat{\bm{n}}}        \newcommand{\nh}{\hat{n}}
\newcommand{\vo}{\bm{o}}       \newcommand{\voh}{\hat{\bm{o}}}        \newcommand{\oh}{\hat{o}}
\newcommand{\vp}{\bm{p}}       \newcommand{\vph}{\hat{\bm{p}}}        \newcommand{\ph}{\hat{p}}
\newcommand{\vq}{\bm{q}}       \newcommand{\vqh}{\hat{\bm{q}}}        \newcommand{\qh}{\hat{q}}
\newcommand{\vr}{\bm{r}}       \newcommand{\vrh}{\hat{\bm{r}}}        \newcommand{\rh}{\hat{r}}
\newcommand{\vs}{\bm{s}}       \newcommand{\vsh}{\hat{\bm{s}}}        \newcommand{\sh}{\hat{s}}
\newcommand{\vt}{\bm{t}}       \newcommand{\vth}{\hat{\bm{t}}}        \newcommand{\that}{\hat{t}}
\newcommand{\vu}{\bm{u}}       \newcommand{\vuh}{\hat{\bm{u}}}        \newcommand{\uh}{\hat{u}}
\newcommand{\vv}{\bm{v}}       \newcommand{\vvh}{\hat{\bm{v}}}        \newcommand{\vh}{\hat{v}}
\newcommand{\vw}{\bm{w}}       \newcommand{\vwh}{\hat{\bm{w}}}        \newcommand{\wh}{\hat{w}}
\newcommand{\vx}{\bm{x}}       \newcommand{\vxh}{\hat{\bm{x}}}        \newcommand{\xh}{\hat{x}}
\newcommand{\vy}{\bm{y}}       \newcommand{\vyh}{\hat{\bm{y}}}        \newcommand{\yh}{\hat{y}}
\newcommand{\vz}{\bm{z}}       \newcommand{\vzh}{\hat{\bm{z}}}        \newcommand{\zh}{\hat{z}}

% %%%%%%%%%%%%%%%%%%%% BOLD GREEK %%%%%%%%%%%%%%%%%%%%%%%%%%%%% 
% Same convention as for ordinary roman letters above
\newcommand{\valpha}  {\bm{\alpha}}      \newcommand{\valphah}  {\hat{\bm{\alpha}}}   
\newcommand{\vbeta}   {\bm{\beta}}       \newcommand{\vbetah}   {\hat{\bm{\beta}}}    
\newcommand{\vdelta}  {\bm{\delta}}      \newcommand{\vdeltah}  {\hat{\bm{\delta}}}   
\newcommand{\vepsilon}{\bm{\epsilon}}    \newcommand{\vepsilonh}{\hat{\bm{\epsilon}}} 
\newcommand{\vphi}    {\bm{\phi}}        \newcommand{\vphih}    {\hat{\bm{\phi}}}     
\newcommand{\vgamma}  {\bm{\gamma}}      \newcommand{\vgammah}  {\hat{\bm{\gamma}}}   
\newcommand{\veta}    {\bm{\eta}}        \newcommand{\vetah}    {\hat{\bm{\eta}}}     
\newcommand{\vtheta}  {\bm{\theta}}      \newcommand{\vthetah}  {\hat{\bm{\theta}}}   
\newcommand{\vkappa}  {\bm{\kappa}}      \newcommand{\vkappah}  {\hat{\bm{\kappa}}}   
\newcommand{\vlambda} {\bm{\lambda}}     \newcommand{\vlambdah} {\hat{\bm{\lambda}}}  
\newcommand{\vmu}     {\bm{\mu}}         \newcommand{\vmuh}     {\hat{\bm{\mu}}}      
\newcommand{\vnu}     {\bm{\nu}}         \newcommand{\vnuh}     {\hat{\bm{\nu}}}      
\newcommand{\vpi}     {\bm{\pi}}         \newcommand{\vpih}     {\hat{\bm{\pi}}}      
\newcommand{\vchi}    {\bm{\chi}}        \newcommand{\vchih}    {\hat{\bm{\chi}}}     
\newcommand{\vrho}    {\bm{\rho}}        \newcommand{\vrhoh}    {\hat{\bm{\rho}}}     
\newcommand{\vsigma}  {\bm{\sigma}}      \newcommand{\vsigmah}  {\hat{\bm{\sigma}}}   
\newcommand{\vtau}    {\bm{\tau}}        \newcommand{\vtauh}    {\hat{\bm{\tau}}}     
\newcommand{\vupsilon}{\bm{\upsilon}}    \newcommand{\vupsilonh}{\hat{\bm{\upsilon}}} 
\newcommand{\vomega}  {\bm{\omega}}      \newcommand{\vomegah}  {\hat{\bm{\omega}}}   
\newcommand{\vxi}     {\bm{\xi}}         \newcommand{\vxih}     {\hat{\bm{\xi}}}      
\newcommand{\vpsi}    {\bm{\psi}}        \newcommand{\vpsih}    {\hat{\bm{\psi}}}     
\newcommand{\vzeta}   {\bm{\zeta}}       \newcommand{\vzetah}   {\hat{\bm{\zeta}}}

% %%%%%%%%%%%%%  MATRICES %%%%%%%%%%%%%%%%%%%%%%%%%%%%%  
\newcommand{\ma}{\bm{A}}       \newcommand{\mah}{\hat{\bm{A}}}   
\newcommand{\mb}{\bm{B}}       \newcommand{\mbh}{\hat{\bm{B}}}   
\newcommand{\mc}{\bm{C}}       \newcommand{\mch}{\hat{\bm{C}}}   
\newcommand{\md}{\bm{D}}       \newcommand{\mdh}{\hat{\bm{D}}}   
\newcommand{\me}{\bm{E}}       \newcommand{\meh}{\hat{\bm{E}}}   
\newcommand{\mf}{\bm{F}}       \newcommand{\mfh}{\hat{\bm{F}}}   
\newcommand{\mg}{\bm{G}}       \newcommand{\mgh}{\hat{\bm{G}}}   
\newcommand{\mH}{\bm{H}}       \newcommand{\mhh}{\hat{\bm{H}}}   
\newcommand{\mi}{\bm{I}}       \newcommand{\mih}{\hat{\bm{I}}}   
\newcommand{\mj}{\bm{J}}       \newcommand{\mjh}{\hat{\bm{J}}}   
\newcommand{\mk}{\bm{K}}       \newcommand{\mkh}{\hat{\bm{K}}}   
\newcommand{\ml}{\bm{L}}       \newcommand{\mlh}{\hat{\bm{L}}}   
\newcommand{\mm}{\bm{M}}       \newcommand{\mmh}{\hat{\bm{M}}}   
\newcommand{\mn}{\bm{N}}       \newcommand{\mnh}{\hat{\bm{N}}}   
\newcommand{\mo}{\bm{O}}       \newcommand{\moh}{\hat{\bm{O}}}   
\newcommand{\mP}{\bm{P}}       \newcommand{\mph}{\hat{\bm{P}}}   
\newcommand{\mq}{\bm{Q}}       \newcommand{\mqh}{\hat{\bm{Q}}}   
\newcommand{\mr}{\bm{R}}       \newcommand{\mrh}{\hat{\bm{R}}}   
\newcommand{\ms}{\bm{S}}       \newcommand{\msh}{\hat{\bm{S}}}   
\newcommand{\mT}{\bm{T}}       \newcommand{\mth}{\hat{\bm{T}}}   
\newcommand{\mU}{\bm{U}}       \newcommand{\muh}{\hat{\bm{U}}}   
\newcommand{\mv}{\bm{V}}       \newcommand{\mvh}{\hat{\bm{V}}}   
\newcommand{\mw}{\bm{W}}       \newcommand{\mwh}{\hat{\bm{W}}}   
\newcommand{\mx}{\bm{X}}       \newcommand{\mxh}{\hat{\bm{X}}}   
\newcommand{\my}{\bm{Y}}       \newcommand{\myh}{\hat{\bm{Y}}}   
\newcommand{\mz}{\bm{Z}}       \newcommand{\mzh}{\hat{\bm{Z}}}   

%%%%%%%%%%%%%%%%%%%%%%%%% CALLIGRAPHIC LETTERS %%%%%%%%%%%%%%%%%%%%%
\newcommand{\ac}{\mathcal{a}}    \newcommand{\Ac}{\mathcal{A}}  
\newcommand{\bc}{\mathcal{b}}    \newcommand{\Bc}{\mathcal{B}}  
\newcommand{\cc}{\mathcal{c}}    \newcommand{\Cc}{\mathcal{C}}  
\newcommand{\dc}{\mathcal{d}}    \newcommand{\Dc}{\mathcal{D}}  
\newcommand{\ec}{\mathcal{e}}    \newcommand{\Ec}{\mathcal{E}}  
\newcommand{\fc}{\mathcal{f}}    \newcommand{\Fc}{\mathcal{F}}  
\newcommand{\gc}{\mathcal{g}}    \newcommand{\Gc}{\mathcal{G}}  
\newcommand{\hc}{\mathcal{h}}    \newcommand{\Hc}{\mathcal{H}}  
\newcommand{\ic}{\mathcal{i}}    \newcommand{\Ic}{\mathcal{I}}  
\newcommand{\jc}{\mathcal{j}}    \newcommand{\Jc}{\mathcal{J}}  
\newcommand{\kc}{\mathcal{k}}    \newcommand{\Kc}{\mathcal{K}}  
\newcommand{\lc}{\mathcal{l}}    \newcommand{\Lc}{\mathcal{L}}  
\newcommand{\mcal}{\mathcal{m}}  \newcommand{\Mc}{\mathcal{M}}  
\newcommand{\nc}{\mathcal{n}}    \newcommand{\Nc}{\mathcal{N}}  
\newcommand{\oc}{\mathcal{o}}    \newcommand{\Oc}{\mathcal{O}}  
\newcommand{\pc}{\mathcal{p}}    \newcommand{\Pc}{\mathcal{P}}  
\newcommand{\qc}{\mathcal{q}}    \newcommand{\Qc}{\mathcal{Q}}  
\newcommand{\rc}{\mathcal{r}}    \newcommand{\Rc}{\mathcal{R}}  
\newcommand{\scal}{\mathcal{s}}  \newcommand{\Sc}{\mathcal{S}}  
\newcommand{\tc}{\mathcal{t}}    \newcommand{\Tc}{\mathcal{T}}  
\newcommand{\uc}{\mathcal{u}}    \newcommand{\Uc}{\mathcal{U}}  
\newcommand{\vcal}{\mathcal{v}}  \newcommand{\Vc}{\mathcal{V}}  
\newcommand{\wc}{\mathcal{w}}    \newcommand{\Wc}{\mathcal{W}}  
\newcommand{\xc}{\mathcal{x}}    \newcommand{\Xc}{\mathcal{X}}  
\newcommand{\yc}{\mathcal{y}}    \newcommand{\Yc}{\mathcal{Y}}  
\newcommand{\zc}{\mathcal{z}}    \newcommand{\Zc}{\mathcal{Z}}  

% ================= NORMS ===========================
\newcommand{\mynorm}[2]{\| {#1} \|_{#2}}
\newcommand{\norm}[2]{\mynorm{#1}{#2}}
\newcommand{\bignorm}[2]{\left\| {#1} \right\|_{#2}}

\newcommand{\onenorm}[1]{\mynorm{#1}{1}}
\newcommand{\bigonenorm}[1]{\bignorm{#1}{1}}

\newcommand{\infnorm}[1]{\mynorm{#1}{\infty}}
\newcommand{\biginfnorm}[1]{\bignorm{#1}{\infty}}

\newcommand{\oneinf}{\ell_{1,\infty}}
\newcommand{\onetwo}{\ell_{1,2}}
\newcommand{\oneinfnorm}[1]{\mynorm{#1}{1,\infty}}
\newcommand{\bigoneinfnorm}[1]{\bignorm{#1}{1,\infty}}

\newcommand{\onetwonorm}[1]{\mynorm{#1}{1,2}}
\newcommand{\bigonetwonorm}[1]{\bignorm{#1}{1,2}}

\newcommand{\twonorm}[1]{\mynorm{#1}{2}}
\newcommand{\bigtwonorm}[1]{\bignorm{#1}{2}}

\newcommand{\znorm}[1]{\mynorm{#1}{0}}
\newcommand{\bigznorm}[1]{\bignorm{#1}{0}}

\newcommand{\frob}[1]{\|{#1}\|_{\text{F}}}
\newcommand{\bigfrob}[1]{\bignorm{#1}{\text{F}}}


% ================ SUMs, INTEGRALS, WITHOUT LIMITS ================
\newcommand{\nlsum}{\sum\nolimits}
\newcommand{\nlprod}{\prod\nolimits}
\newcommand{\nlint}{\int\nolimits}
\newcommand{\nlmin}{\min\nolimits}


% =============== USEFUL SETS, FIELDS, ETC. ===========================
\newcommand{\R}{\mathbb{R}}
\newcommand{\N}{\mathbb{N}}
\newcommand{\reals}{\mathbb{R}}
\newcommand{\complex}{\matbb{C}}
\newcommand{\integers}{\mathbb{Z}}
\newcommand{\posdef}[1]{S_{++}^{#1}}
\newcommand{\semidef}[1]{S_+^{#1}}

% =============== MISC CONSTANTS ============================
\newcommand{\half}{\tfrac{1}{2}}
\newcommand{\third}{\tfrac{1}{3}}
\newcommand{\fourth}{\tfrac{1}{4}}
\newcommand{\onehalf}{\tfrac{3}{2}}


% %%%%%%%%%%% MATH KEYWORDS %%%%%%%%%%%%%%%%%%%%%%%%%%
\DeclareMathOperator*{\argmin}{argmin}
\DeclareMathOperator*{\argmax}{argmax}
\DeclareMathOperator{\dom}{dom}
\DeclareMathOperator{\interior}{int}
\DeclareMathOperator{\rank}{rank}
\DeclareMathOperator{\ri}{ri}
\DeclareMathOperator{\sgn}{sgn}
\DeclareMathOperator{\trace}{Tr}
\DeclareMathOperator{\Diag}{Diag}
\DeclareMathOperator{\range}{range}
\DeclareMathOperator{\vect}{vec}
\DeclareMathOperator{\prox}{prox}
\DeclareMathOperator{\intr}{int}
\DeclareMathOperator{\relint}{ri}


% SS: 6th June, 2011 --- I have upper-cased everything to match the style used
% in the book; lower case references are not to be used in the book; if they
% are, those are the typos!

% === References to figures, theorems, etc. ===
\newcommand{\prop}[1]{Proposition~\protect\ref{#1}}
\newcommand{\coro}[1]{Corollary~\protect\ref{#1}}
\newcommand{\rema}[1]{Remark~\protect\ref{#1}}
\newcommand{\exam}[1]{Example~\protect\ref{#1}}
\newcommand{\appl}[1]{Application~\protect\ref{#1}}
% [SN]: lowercase \algo, uppercase \Algo
\newcommand{\Algo}[1]{Algorithm~\protect\ref{#1}}
\newcommand{\algo}[1]{Algorithm~\protect\ref{#1}}
\newcommand{\opti}[1]{Optimization Problem~\protect\ref{#1}}
\newcommand{\defi}[1]{Definition~\protect\ref{#1}}
\newcommand{\prob}[1]{Problem~\protect\ref{#1}}
% [SN]: lowercase \theo, uppercase \Theo
\newcommand{\Theo}[1]{Theorem~\protect\ref{#1}}
\newcommand{\theo}[1]{Theorem~\protect\ref{#1}}
% [SN]: lowercase \fig, uppcase \Fig
\newcommand{\Fig}[1]{Figure~\protect\ref{#1}}
\newcommand{\fig}[1]{Figure~\protect\ref{#1}}
\newcommand{\tab}[1]{Table~\protect\ref{#1}}
\newcommand{\chap}[1]{Chapter~\protect\ref{#1}}
% [SN]: lowercase \sec, uppcase \Sec
% TODO: check all occurences of \sec at the beginning of a sentence
\newcommand{\Sec}[1]{Section~\protect\ref{#1}}
\renewcommand{\sec}[1]{Section~\protect\ref{#1}} % SS: but why?????? it shd always be uppercase!
\newcommand{\app}[1]{Appendix~\protect\ref{#1}}
\newcommand{\eq}[1]{(\protect\ref{#1})}
\newcommand{\pag}[1]{page \protect\pageref{#1}}

% === Marginal notes ===
% To use marginal notes, type in \margin{this goes into the margin}
\newcommand{\margin}[1]{\marginpar[\flushleft{#1}]{right}}



\newcommand{\vxp}{\vx^+}
\newcommand{\vxn}{\vx^-}
\newcommand{\dataset}{{\cal D}}
\newcommand{\fracpartial}[2]{\frac{\partial #1}{\partial  #2}}

% Heading arguments are {volume}{year}{pages}{submitted}{published}{author-full-names}

\jmlrheading{1}{2014}{1-48}{4/00}{10/00}{Name 1 and Name2}

% Short headings should be running head and authors last names

\ShortHeadings{Paired passive aggressive for ranking and classification}{Name1 and Name2}
\firstpageno{1}

\begin{document}

\title{Paired passive aggressive for ranking and classification}

\author{       \name Author I \email author1@somewhere \\
       \addr Department of \\
       University of \\
       City, WA 98195-4322, USA
       \AND
       \name Author II \email author2@somewhere \\
       \addr Department of \\
       University of \\
       City, WA 98195-4322, USA
       }
\editor{some editor}

\maketitle

\begin{abstract}%   <- trailing '%' for backward compatibility of .sty file
This paper describes the ....
\end{abstract}

\begin{keywords}
  Passive aggressive, AUC, MAP
\end{keywords}


\section{Introduction}

The importance of the problem

Related work

\section{Problem Setting}

In this section we introduce the notation used throughout the paper and describe our problem setting. Vectors are denoted by lower case bold face letters (e.g. $\vx$ and $\vw$) 
%where the $i^{th}$ element of the vector $\vx$ is denoted by $\vx_i$. 
The hinge function is denoted by $[x]_+ = max\{0, x\} $. Sets of indices are denoted by capital curly letters (e.g. $\mathcal{J}$). We denote samples which arrive from the $k$ class using superscript (e.g. $\vx^k$). Subscript will denote the time point the samples is introduced (e.g. $\vx^k_t$ is a vector from the $k$ class at time $t$). \\

We are interested in the case where at each time point $t$ we receive a batch of $k_t$ sample and than choose how to update the vector weights $\vw$. Specificly we are interest in the case where each of the $k_t$ samples arrives from a different class. At each time point $t$ we solve an optimization problem which performs a trade off between two things. First, it aims that the new solution $\vw$ will be close to the former weight vector $w_t$. Second we prefer to classify all the samples provided at time $t$ correctly with a margin of 1.

\begin{equation*}
\begin{aligned}
& \underset{\vw}{\text{minimize}}
& & \frac{1}{2} || \vw - \vw_t ||^2 + C \sum\limits_{k=1}^{k_t}{\xi^k} \\
& \text{subject to}
& & 1 - \vw^T \vx^k y^k \leq \xi^k, \;
 \; k = 1, \ldots, k_t.
\end{aligned}
\end{equation*}

There are some benefits of an update which uses several samples for the update. First, in cases where the data is unbalanced, using a balanced updating scheme that introduce an equal number of samples at time point $t$ we can come up with guaranties both for the classification mistakes and for the AUC.
Second, an update rule that uses several samples at a single time point $t$ is internally tuned since the we need to advanced $\vw$ in a way that is agreeable with the samples at time $t$. In a way the other classes are controling the step size that is made. Much like the case in multibatch stochastic gradient descent, one of the benfits is that the steps are more moderate and there is less variablity in each small step. Only here we are using the variablity between different classes and not the variablity within the class.
\section{Average Classification Loss}

Define the loss and the problem

Derive the update rule


Theorem 1: the expected loss is less than the average loss

Theorem 2: 1-AUC is bounded 

Theorem 3: Show that classification is correct and $\vxp >0$ while $\vxn < 0$ after the update - this is not correct since we are balancing with $w_t$ from the former step.
This would have been correct if we would have possed this is a feasibility problem (perceptron style).

\section{Update rules}

We are interested in solving a passive aggressive style problem only that we are shown $k_t$ examples at time $t$. Specifically, we are interested in the case that the samples arrive from different classes. We show that by using a balanced regiem we can provide bounds for the AUC and for a multiclass AUC. The specific method we choose to optimize the problem have a great deal of implication on the solution where different steps will yield different results.

\begin{equation*}
\begin{aligned}
& \underset{\vw}{\text{minimize}}
& & \frac{1}{2} || \vw - \vw_t ||^2 + C \sum\limits_{k=1}^{k_t}{\xi^k} \\
& \text{subject to}
& & 1 - \vw^T \vx^k \vy^k \leq \xi^k, \;
 \; k = 1, \ldots, k_t.
\end{aligned}
\end{equation*}


The dual problem is 
\begin{equation*}
\begin{aligned}
& \underset{\valpha}{\text{maximize}}
& & \frac{1}{2} || \sum\limits_{i=1}^{k_t} {\alpha_i \vx^k y^k}||^2 + \sum\limits_{k=1}^{k_t} {\alpha_i (1 - \vw_t^T \vx^k \vy^k }) \\
& \text{subject to}
& & 0 \leq \alpha_i \leq C, \;
 \; k = 1, \ldots, k_t.
\end{aligned}
\end{equation*}


This problem does not have a closed analytic solution, So we aim to maximize the dual function iterativly. One way of solving this is using ${\i dual coordinate ascent}$ (DCA) on all the $k_t$ samples. DCA will make iteration until convergence. However as noticed by ...,  we can also make a small advancement that is not optimal but that will advance us in the right direction. For example the passive aggressive algorithm can be thought of as single iteration of DCA where we update each dual variable only once. We suggest that this can be extended by advancing with a step that is made using the joined information of several samples. We choose a set of indices $\mathcal{J} $ to increase using the ${\b same }$ step.

\[ \alpha_j = \alpha_j + \tau , \;\; j \in \mathcal{J} \]
We derive the following $\tau$
\[ \tau = max(L_b   , min(U_b , \frac{|\mathcal{J}| - \vw_t^T \sum\limits_{j \in \mathcal{J}} {\vx^j y^j}   }{|| \sum\limits_{j \in \mathcal{J} }{\vx^j y^j} ||^2 }  ) )
\]
Where $ L_b = max_{ j \in \mathcal{J} } (-\alpha_j) $ and $ U_b = min_{ j \in \mathcal{J} } (C -\alpha_j) $ which appear since each of the updated $\alpha_j$ needs to keep its constrain $0 \leq \alpha_j \leq C $.
It is possible that a step that advances all $j \in \mathcal{J} $ does not exists because $ U_b $ could be smaller than $L_b$. We will always be able to perform at least one such step since we initialize $\alpha_i$ as zero. If we partition the set of samples and at each iteration we use a different partition, we are guaranteed that $L_b \leq U_b $ since we advance all the dual variables in each partition with the same steps.\\



We can think of this as updating a new vector $\vx^\mathcal{J} = \sum\limits_{j \in \mathcal{J} }{\vx_j y_j}$ where $y^\mathcal{J} = 1 $. Only here $\vx^\mathcal{J}$ should be correct with a margin of $|\mathcal{J}| $.  We than update $\vw$ using $\vx^\mathcal{J}$ with the step size $\tau$. This analogy is not entirely correct because of the constrains we have on the $\tau$.


Notice that by using multiple items in $\mathcal{J}$ we make a statement about their linear combination and not any of them individually. For example, when we update two items a positive sample $\vxp$ and a negative sample $\vxn$ forcing that their sum should be classified positive $0 \leq \vw^T (\vxp - \vxn) $ we actually argue about their order we say that their difference should be kept positive or that $ \vw^T \vxn \leq \vw^T \vxp$.


Using various sets of $ \mathcal{J} $ and various number of iteration at time $t$ we propose several update rules:\\

In the case where at time $t$ we are provided with two samples ($k_t=2$), $\vxp, \vxn$.\\
I. PA-DCA - Iterate until convergence at each iteration choose a single sample ($|\mathcal{J}|=1$).\\
II. PA-sequential - Iterate only once for $\vxp$ and than once for $\vxn$.\\
III. PA-AUC - Iterate only once using both samples $\mathcal{J} = \{\vxp, \vxn\} $ .\\
IV. PA-correctMistakes - Iterate only once use only the samples that failed to achieve correct classification with the margin. $\mathcal{J} = \{\vxp, \vxn\} \; or \;\{\vxp \} \; or \; \{ \vxn \}$.\\

In the case where we are presented $k$ samples. \\
I. PA-DCA - Iterate until convergence at each iteration choose a single sample ($|\mathcal{J}|=1$).\\
II. PA-sequential - Iterate only once for each sample.\\
III. PA-maxViolators - Iterate only once. Here $\mathcal{J}$ contains the positive sample that caused the highest loss and the negative sample that caused the highest loss. \\
IV. PA-correctMistakes - Iterate only once use only the samples that failed to achieve correct classification with the margin. $\mathcal{J} = \{ i | 0 \leq l_{w_t}(\vx_i, y_i) \}$.\\


Theorem 4: convergence of DCA

Theorem 5: classification errors --> number of mistakes

Theorem 6: As in the case of the classical passive aggressive if our step is not caped by C after the update we will correctly classify the samples. In case where we are caped by C the loss of the samples we choose to update will decrease. But we are not guarantied a correct classification. \\
Show that classification is correct and $ w x+ >0$ while $w x- < 0$ after the update when it is not caped.

Theorem 7: from Theorem 5 it follows that 1-AUC is bounded

\section{Calibrated Multilabel Classification and Ranking}

Calibrated separation ranking loss was proposed by \cite{GuoShuurmans11}

We are interested in the case where we have more than 2 classses and these classes are unbalanced. For this multliclass scenario we follow the $mutliclass AUC$ suggested by ... 
We define $AUC_{all pairs}$ by:
\[
	AUC_{all pairs} = \frac{1}{K} \sum\limits_{k=1}^{K}  \frac{1}{K-1}\sum\limits_{l \neq k} (AUC_{w^k}(C^k, C^l)) 
\]
Where $C_i$ denote all the samples from class $i$, and $AUC_{w^k}(C^k, C^l)$ is the AUC performed on the samples from the $k$ class and samples from the $l$ class using the classifier $ \vw^k $ train using the samples from the k class as positivies.
\[
1 - AUC_{all pairs} \leq \frac{1}{K} ( \sum\limits_{k=1}^{K} E(M_{w^k}(C^k)) + \frac{1}{k-1} ( \sum\limits_{l \neq k}  E(M_{w^k}(C^l)) )
\]
This suggest that at time $t$ to train our classifier $\vw^k $ we need to present to it a positive sample from the $k$ class (with a weight of K-1) and negative samples from the other K-1 classes.\\
\\
\[
  \vw^k_t = \vw^k_{t-1} + (K-1)\alpha^k_t \vx^k_t - \sum\limits_{l \neq k} \alpha^l_t \vx^l_t
\]
Derive update rule

Theorem 8: correct multilabels are above the incorrect set of labels

Can we show that 1-MAP is bounded?



\section{Experiments}

Synthetic data

Discriminative keyword spotting with algo 1 and algo 2

Mutliclass classification evaluated using $AUC_{all \; pairs}$ and $AUC_{one \; vs \; all}$

LETOR3 data for ranking with algo 1 and algo 2

multi label - Reuters


% Acknowledgements should go at the end, before appendices and references

\acks{We would like to acknowledge support for this project
.... }

% Manual newpage inserted to improve layout of sample file - not
% needed in general before appendices/bibliography.

\newpage

\appendix
\section*{Appendix A.}
\label{app:theorem}

% Note: in this sample, the section number is hard-coded in. Following
% proper LaTeX conventions, it should properly be coded as a reference:

%In this appendix we prove the following theorem from
%Section~\ref{sec:textree-generalization}:

In this appendix we prove the following theorem from
Section~X.X:

\noindent

{\bf Theorem} {\it First we will show that the 1 - AUC is bounded by the mean 
of the errors in the first class and the mean number of errors in the second
class.  $1 - AUC \leq E(M^+) + E(M^-)$ 
} \hfill\BlackBox

{\bf Proof} 
\begin{multline}
1 - AUC = \frac{1}{|X^-| |X^+|} \sum\limits_{x^+ \in X^+, x^- \in X^-}{  \mathbbm{1}_{\vw^T \vx^+ \leq \vw^T \vx^-} } \;\;\leq \\
\frac{1}{|X^-| |X^+|}  \sum\limits_{\vxp \in X^+, \vxn \in X^-}{  \mathbbm{1}_{\vw^T \vxp \leq 0} \;+\; \mathbbm{1}_{0 \leq \vw^T \vxn}  }  \;\;= \\
\frac{1}{|X^+|}  \sum\limits_{\vxp \in X^+}{  \mathbbm{1}_{\vw^T \vxp \leq 0} } \;+\; \frac{1}{|X^-|}  \sum\limits_{\vxn \in X^-}{\mathbbm{1}_{0 \leq \vw^T \vxn }  }   \;= \;
E(M^+) + E(M^-)
\end{multline}
and that is that.\\


{\bf Theorem} {\it The sum of the average mistake in the two classes can be bounded \\
$E(M^+) + E(M^-) \leq 
\frac{1}{|X^-| |X^+|} \; max\{R^2, 1/C\}( ||{\bf u}||^2 + 2C \sum\limits_{t=1}^T{ l_t^{*+} + l_t^{*-}} ) $
} \hfill\BlackBox

{\bf Proof}.
\begin{multline}
.
\end{multline}


\vskip 0.2in
\bibliography{sample}

\end{document}
