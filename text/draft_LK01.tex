\documentclass[twoside,11pt]{article}

% Any additional packages needed should be included after jmlr2e.
% Note that jmlr2e.sty includes epsfig, amssymb, natbib and graphicx,
% and defines many common macros, such as 'proof' and 'example'.
%
% It also sets the bibliographystyle to plainnat; for more information on
% natbib citation styles, see the natbib documentation, a copy of which
% is archived at http://www.jmlr.org/format/natbib.pdf

\usepackage{jmlr2e}
\usepackage{bbm}
\usepackage{amsmath}

% Definitions of handy macros can go here

\newcommand{\dataset}{{\cal D}}
\newcommand{\fracpartial}[2]{\frac{\partial #1}{\partial  #2}}

% Heading arguments are {volume}{year}{pages}{submitted}{published}{author-full-names}

\jmlrheading{1}{2014}{1-48}{4/00}{10/00}{Name 1 and Name2}

% Short headings should be running head and authors last names

\ShortHeadings{Paired passive aggressive for ranking and classification}{Name1 and Name2}
\firstpageno{1}

\begin{document}

\title{Paired passive aggressive for ranking and classification}

\author{       \name Author I \email author1@somewhere \\
       \addr Department of \\
       University of \\
       City, WA 98195-4322, USA
       \AND
       \name Author II \email author2@somewhere \\
       \addr Department of \\
       University of \\
       City, WA 98195-4322, USA
       }
\editor{some editor}

\maketitle

\begin{abstract}%   <- trailing '%' for backward compatibility of .sty file
This paper describes the ....
\end{abstract}

\begin{keywords}
  Passive aggressive, AUC, MAP
\end{keywords}

\section{Introduction}

Here is some introduction...\\


\section{The second section }
Varient 1 - sequantial balanced passive aggressive \\
Varient 2 - using the different cases \\
Varient 3 - solving the exact using DCA \\
Varient 4 - multiclass \\

{\noindent \em Remainder omitted in this sample. See http://www.jmlr.org/papers/ for full paper.}

% Acknowledgements should go at the end, before appendices and references

\acks{We would like to acknowledge support for this project
.... }

% Manual newpage inserted to improve layout of sample file - not
% needed in general before appendices/bibliography.

\newpage

\appendix
\section*{Appendix A.}
\label{app:theorem}

% Note: in this sample, the section number is hard-coded in. Following
% proper LaTeX conventions, it should properly be coded as a reference:

%In this appendix we prove the following theorem from
%Section~\ref{sec:textree-generalization}:

In this appendix we prove the following theorem from
Section~X.X:

\noindent

{\bf Theorem} {\it First we will show that the 1 - AUC is bounded by the mean 
of the errors in the first class and the mean number of errors in the second
class.  $1 - AUC \leq E(M^+) + E(M^-)$ 
} \hfill\BlackBox

{\bf Proof} 
\begin{multline}
1 - AUC = \frac{1}{|X^-| |X^+|} \sum\limits_{x^+ \in X^+, x^- \in X^-}{  \mathbbm{1}_{w^T x^+ \leq w^T x^-} } \;\;\leq \\
\frac{1}{|X^-| |X^+|}  \sum\limits_{x^+ \in X^+, x^- \in X^-}{  \mathbbm{1}_{w^T x^+ \leq 0} \;+\; \mathbbm{1}_{0 \leq w^T x^-}  }  \;\;= \\
\frac{1}{|X^+|}  \sum\limits_{x^+ \in X^+}{  \mathbbm{1}_{w^T x^+ \leq 0} } \;+\; \frac{1}{|X^-|}  \sum\limits_{x^- \in X^-}{\mathbbm{1}_{0 \leq w^T x^-}  }   \;= \;
E(M^+) + E(M^-)
\end{multline}
and that is that.\\


{\bf Theorem} {\it The sum of the average mistake in the two classes can be bounded \\
$E(M^+) + E(M^-) \leq 
\frac{1}{|X^-| |X^+|} \; max\{R^2, 1/C\}( ||{\bf u}||^2 + 2C \sum\limits_{t=1}^T{ l_t^{*+} + l_t^{*-}} ) $
} \hfill\BlackBox

{\bf Proof}.
\begin{multline}
.
\end{multline}


\vskip 0.2in
\bibliography{sample}

\end{document}
